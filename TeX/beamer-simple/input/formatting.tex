\section{Formatting}

% -----------------------------------------------------------------------

\begin{frame}
  \frametitle{Formatting guidelines}
    \begin{itemize}
    \item Use a 4$\times$3 aspect ratio for older projectors
    \item Use \large{large} \normalsize text for the body and plots
  \item Body text on a screen is most readable when it is sans-serif,
    but also use a standard serif font (e.g. Palatino) which has rich
    math support for math equations
    \begin{itemize}
    \item Roboto for body text
    \item Palatino for math
    \item Inconsolata for \texttt{fixed width} text.
    \end{itemize}
  \end{itemize}
  \cite*{horseshoe}
\end{frame}

% -----------------------------------------------------------------------

\begin{frame}
  \frametitle{Preview of font appearances}

  The density of the univariate Gaussian random variable denoted by
  $x \sim \mathcal{N}(\mu, \sigma^2)$ is given by
  $f(x; \mu, \sigma^2)$, for location parameter $\mu$ and scale
  parameter $\sigma > 0$,
%
  \begin{align*}
    f(x; \mu, \sigma^2) &= \frac{1}{\sqrt{2\pi\sigma^2}} \exp 
           \left[
           -\frac{1}{2\sigma^2}(x - \mu)^2
           \right]
  \end{align*}
%
  \textbf{Generally, there is also the multivariate Gaussian}
  $\mathbf{x} \sim \mathcal{N}_p(\mathbf{m}, \Sigma)$. \emph{The maximum
  likelihood estimate is} {\textmu}m. 
  %
  \begin{align*}
    (\hat{\mu}, \hat{\sigma}^2)
    &= \arg \max_{(\mu, \sigma) \in \mathbb{R} \times \mathbb{R}^+} \prod_{i=1}^N f(x_i; \mu, \sigma^2).
  \end{align*}
  
\end{frame}

% -----------------------------------------------------------------------


%%% Local Variables:
%%% mode: latex
%%% TeX-master: "../main"
%%% End:
