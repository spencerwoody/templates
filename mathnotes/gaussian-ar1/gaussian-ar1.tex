
% -------------------------------------------------------------------------
% Preamble

\documentclass[letter, 10pt]{article}

% Fonts
\usepackage{mathpazo}
\usepackage{inconsolata}

% Basics
\usepackage{setspace}
\usepackage{graphicx} % including images
\usepackage{amsmath} 
\usepackage{amssymb}
\usepackage{subcaption}
\usepackage{enumerate}
\usepackage{booktabs}

\usepackage{authblk}            % Multiple authors
\renewcommand\Authands{ and }

% Bibliography format
\usepackage[round]{natbib} 
\bibliographystyle{plainnat}

% For using Computer Modern version of blackboard symbols
\AtBeginDocument{
  \DeclareSymbolFont{AMSb}{U}{msb}{m}{n}
  \DeclareSymbolFontAlphabet{\mathbb}{AMSb}}

% Math shortcuts

% Distributions
\newcommand{\N}{\mathcal{N}}    %Gaussian distribution
\newcommand{\IG}{\mathcal{IG}}    %Inverse gamma
\newcommand{\G}{\mathcal{G}}    %Gamma

% Variance, expectation, covariance
\newcommand{\var}{\mathrm{var}} 
\newcommand{\Var}{\mathrm{Var}}
\newcommand{\cov}{\mathrm{cov}}
\newcommand{\Cov}{\mathrm{Cov}}
\newcommand{\E}{\mathrm{E}}

% Matrices: zeros, identity
\newcommand{\Oh}{\mathcal{O}}
\newcommand{\I}{\mathcal{I}}

% Independence symbol
\newcommand\ind{\protect\mathpalette{\protect\independenT}{\perp}}
\def\independenT#1#2{\mathrel{\rlap{$#1#2$}\mkern2mu{#1#2}}}



% -------------------------------------------------------------------------


\title{Gaussian AR(1) model}
\author{Spencer Woody}
\date{\today}

% -------------------------------------------------------------------------
% Begin document

\begin{document}

\maketitle

% -------------------------------------------------------------------------
% BODY

\paragraph{Model}

For a fixed $\phi$ and $\tau^2$,
\begin{align*}
  x_t \mid x_{t - 1} &\sim \N(\phi x_{t-1}, \tau^2)
\end{align*}

\subparagraph{Marginal mean}

Assume a stationary process and $\phi \neq 1$. The marginal
expectation of $x_t$ is
\begin{align*}
  m := \E(x_t) &= \E(\E(x_t \mid x_{t-1})) \\
               &= \E(\phi x_{t-1}) \\
               &= \phi m ,
\end{align*}
implying that $m = 0$.

\subparagraph{Marginal variance}

\begin{align*}
  v := \Var(x_t) &= \E(\Var(x_t \mid x_{t - 1})) + \Var(\E(x_t \mid x_{t-1}))\\
                 &= \E(\tau^2) + \Var{\phi x_{t-1}} \\
                 &= \tau^2 + \phi^2 v,
\end{align*}
implying that $v = \tau^2 / (1 - \phi^2)$, and this quantity is
greater than zero only if $|\phi| < 1$.

\subparagraph{Joint distribution}

\begin{align*}
  x_{1:n} \sim \N(0, v\Phi_n) ,
\end{align*}
where $\Phi_n$ is the correlation matrix, and can be defined
element-wise. The $(i, j)$ element of this matrix is,
\begin{align*}
  \{\Phi_n\}_{ij} &= \phi^{|i - j|}.
\end{align*}
The inverse of the correlation matrix, $\Omega_n := \Phi_n^{-1}$ is
sparse (tri-diagonal), making for efficient likelihood evaluations. It
can also be defined element-wise,
\begin{align*}
  \{\Omega_n\}_{ij} = \frac{1}{1 - \phi^2} \cdot
  \begin{cases}
    1 & \text{ if } i = j \\
    -\phi & \text{ if } i = j \pm 1 \\
    0 & \text{ otherwise}
  \end{cases}
\end{align*}

% -------------------------------------------------------------------------
% End document

\end{document}

%%% Local Variables:
%%% mode: latex
%%% TeX-master: t
%%% End:
