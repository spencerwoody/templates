
% -------------------------------------------------------------------------
% Preamble

\documentclass[letter, 10pt]{article}

% Fonts
\usepackage{mathpazo}
\usepackage{inconsolata}

% Basics
\usepackage{setspace}
\usepackage{graphicx} % including images
\usepackage{amsmath} 
\usepackage{amssymb}
\usepackage{subcaption}
\usepackage{enumerate}
\usepackage{booktabs}

\usepackage{authblk}            % Multiple authors
\renewcommand\Authands{ and }

% Bibliography format
\usepackage[round]{natbib} 
\bibliographystyle{plainnat}

% For using Computer Modern version of blackboard symbols
\AtBeginDocument{
  \DeclareSymbolFont{AMSb}{U}{msb}{m}{n}
  \DeclareSymbolFontAlphabet{\mathbb}{AMSb}}

% Math shortcuts

% Distributions
\newcommand{\N}{\mathcal{N}}    %Gaussian distribution
\newcommand{\IG}{\mathcal{IG}}    %Inverse gamma
\newcommand{\G}{\mathcal{G}}    %Gamma

% Variance, expectation, covariance
\newcommand{\var}{\mathrm{var}} 
\newcommand{\Var}{\mathrm{Var}}
\newcommand{\cov}{\mathrm{cov}}
\newcommand{\Cov}{\mathrm{Cov}}
\newcommand{\E}{\mathrm{E}}

% Matrices: zeros, identity
\newcommand{\Oh}{\mathcal{O}}
\newcommand{\I}{\mathcal{I}}

% Independence symbol
\newcommand\ind{\protect\mathpalette{\protect\independenT}{\perp}}
\def\independenT#1#2{\mathrel{\rlap{$#1#2$}\mkern2mu{#1#2}}}



% -------------------------------------------------------------------------


\title{Normal lemma}
\author{Spencer Woody}
\date{\today}

% -------------------------------------------------------------------------
% Begin document

\begin{document}

\maketitle

% -------------------------------------------------------------------------
% BODY

\paragraph{Model}


\begin{align*}
  (y \mid \theta) &\sim \N(R\theta, \Omega) \\
  \theta &\sim \N(m, \Sigma)
\end{align*}

\paragraph{Joint distribution}

Let $\I$ be the identity matrix, and $\Oh$ be a matrix of all
zeros. Furthermore, define the random variable $x \sim \N(0,
\Omega)$. The joint distribution of $(y, \theta)$ can be decomposed as
follows:

\begin{align*}
  \begin{bmatrix}
    y \\ \theta
  \end{bmatrix}
  &=
    \begin{bmatrix}
      R \\ \I
    \end{bmatrix} \theta +
  \begin{bmatrix}
    \I \\ \Oh
  \end{bmatrix} x.
\end{align*}
%
%
From this, the expectation is 
\begin{align*}
  \E 
  \left(
  \begin{bmatrix}
    y \\ \theta
  \end{bmatrix}
  \right)
  &=
    \begin{bmatrix}
      R \\ \I
    \end{bmatrix} m + 0 \\
  &= \begin{bmatrix}
      Rm \\ m
    \end{bmatrix} ,
\end{align*}
and the covariance is
\begin{align*}
  \Cov 
  \left(
  \begin{bmatrix}
    y \\ \theta
  \end{bmatrix}
  \right)
  &=
    \begin{bmatrix}
      R \\ \I
    \end{bmatrix} \Sigma
  \begin{bmatrix}
    R \\ \I
  \end{bmatrix}^T +
  \begin{bmatrix}
    \I \\ \Oh
  \end{bmatrix} \Omega
  \begin{bmatrix}
    \I \\ \Oh
  \end{bmatrix} ^T \\
  &=
    \begin{bmatrix}
      R \Sigma R^T + \Omega & R\Sigma \\
      \Sigma R^T & \Sigma
    \end{bmatrix}.
\end{align*}
%
%
Therefore, the joint distribution of $(y, \theta)$ can be expressed as
%
%
\begin{align*}
  \begin{bmatrix}
    y \\ \theta
  \end{bmatrix}
  &\sim
    \N 
    \left(
    \begin{bmatrix}
      Rm \\ m
    \end{bmatrix} ,
  \begin{bmatrix}
      R \Sigma R^T + \Omega & R\Sigma \\
      \Sigma R^T & \Sigma
    \end{bmatrix}
    \right).
\end{align*}

% -------------------------------------------------------------------------
% End document

\end{document}

%%% Local Variables:
%%% mode: latex
%%% TeX-master: t
%%% End:
