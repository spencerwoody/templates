
% -------------------------------------------------------------------------
% Preamble

\documentclass[letter]{article}

% Fonts

\usepackage{times}
\usepackage{inconsolata}

% Basics
\usepackage{setspace}
\usepackage{graphicx} % including images
\usepackage{amsmath} 
\usepackage{amssymb}
\usepackage{subcaption}
\usepackage{enumerate}
\usepackage{booktabs}


\usepackage{authblk}            % Multiple authors
\renewcommand\Authands{ and }

% Bibliography format
\usepackage[round]{natbib} 
\bibliographystyle{plainnat}

% For using Computer Modern version of blackboard symbols
\AtBeginDocument{
  \DeclareSymbolFont{AMSb}{U}{msb}{m}{n}
  \DeclareSymbolFontAlphabet{\mathbb}{AMSb}}

% Math shortcuts

% Distributions
\newcommand{\N}{\mathcal{N}}    %Gaussian distribution
\newcommand{\IG}{\mathcal{IG}}    %Inverse gamma
\newcommand{\G}{\mathcal{G}}    %Gamma

% Variance, expectation, covariance
\newcommand{\var}{\mathrm{var}} 
\newcommand{\Var}{\mathrm{Var}}
\newcommand{\cov}{\mathrm{cov}}
\newcommand{\Cov}{\mathrm{Cov}}
\newcommand{\E}{\mathrm{E}}

% Matrices: zeros, identity
\newcommand{\Oh}{\mathcal{O}}
\newcommand{\I}{\mathcal{I}}

% Independence symbol
\newcommand\ind{\protect\mathpalette{\protect\independenT}{\perp}}
\def\independenT#1#2{\mathrel{\rlap{$#1#2$}\mkern2mu{#1#2}}}



% -------------------------------------------------------------------------


\title{Some properties of the Gaussian distribution}
\author{Spencer Woody}
\date{\today}

% -------------------------------------------------------------------------
% Begin document

\begin{document}

\maketitle

% -------------------------------------------------------------------------
% BODY

Denote the normal density function
\begin{align*}
  \phi(x) = \frac{1}{\sqrt{2\pi}} \exp \left( - \frac{1}{2} x^2 \right). 
\end{align*}

Its derivative is
\begin{align*}
  \frac{\dee}{\dee x} \phi(x) &=
  \frac{1}{\sqrt{2\pi}} \exp \left( - \frac{1}{2} x^2 \right) \cdot
  (-x) \\
  &= -x  \phi(x)
\end{align*}

The CDF is
\begin{align*}
  \Phi(x) &= \int_{-\infty}^{x} \phi(t) \dee t
\end{align*}
and from the fundamental theorem of calculus, the derivative of the CDF is
\begin{align*}
  \frac{\dee}{\dee x} \Phi(x) = \Phi'(x) = \phi(x)
\end{align*}

The derivative of the inverse CDF is found using the identity shown in Appendix~\ref{sec:deriv-inverse-funct},
\begin{align*}
  \frac{\dee}{\dee p}\Phi^{-1}(p) &= \frac{1}{\Phi'(\Phi^{-1}(p))} \\
                                  &= \frac{1}{\phi(\Phi^{-1}(p))}
\end{align*}


% -------------------------------------------------------------------------
% Appendix

\appendix

\section{Derivative of an inverse function}
\label{sec:deriv-inverse-funct}

For a one-to-one function $f$, the derivative of the inverse function
is found as follows:
\begin{align*}
  f(f^{-1}(x)) &= x \\
   \frac{\dee}{\dee x} f(f^{-1}(x)) &= \frac{\dee}{\dee x} x  \\
   f'(f^{-1}(x)) \frac{\dee}{\dee x} f^{-1}(x) &= 1 \\
  \frac{\dee}{\dee x} f^{-1}(x) &= \frac{1}{f'(f^{-1}(x))}
\end{align*}



% -------------------------------------------------------------------------
% End document

\end{document}

%%% Local Variables:
%%% mode: latex
%%% TeX-master: t
%%% End:
