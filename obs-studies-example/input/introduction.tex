
\section{Introduction}

Psychohistory depends on the idea that, while one cannot foresee the
actions of a particular individual, the laws of statistics as applied
to large groups of people can predict the general flow of future
events~\citep{asimov1951}.  As an analogy, consider a gas.  It is
difficult to predict the motion of a single molecule in a gas, but we
can predict the mass action of the gas to a high level of accuracy.
In this paper, we apply psychohistory to predict the results of
observational studies.  \citet{cochran1965} defined an observational
study as an empiric investigation in which ''...the objective is to
elucidate cause-and-effect relationships…[in which] it is not feasible
to use controlled experimentation, in the sense of being able to
impose the procedures or treatments whose effects it is desired to
discover, or to assign subjects at random to different procedures.''


%%% Local Variables:
%%% mode: latex
%%% TeX-master: "../main"
%%% End:
