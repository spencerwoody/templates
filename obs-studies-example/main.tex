\documentclass[twoside,11pt]{article}

% Any additional packages needed should be included after obs_study_style.
% Note that obs_study_style.sty includes epsfig, amssymb, natbib and graphicx,
% and defines many common macros, such as 'proof' and 'example'.
%
% It also sets the bibliographystyle to plainnat; for more information on
% natbib citation styles, see the natbib documentation, a copy of which

\usepackage{obs_study_style}

% -------------------------------------------------------------------------
% Define macros
% -------------------------------------------------------------------------

\newcommand{\dataset}{{\cal D}}
\newcommand{\fracpartial}[2]{\frac{\partial #1}{\partial  #2}}


% -------------------------------------------------------------------------
% Headings
% -------------------------------------------------------------------------



% For papers submitted for review, just fill in author names For
% accepted papers, heading arguments are
% {volume}{year}{pages}{submitted}{published}{author-full-names}
\heading{}{}{}{}{}{Carlos Carvalho, Jared Murray and Spencer Woody}

% Short headings should be running head and authors last names

\ShortHeadings{Invited Workshop: Empirical Investigation of Methods
  for Heterogeneity}{Carvalho, Murray
  and Woody} \firstpageno{1}

% -------------------------------------------------------------------------
% Begin document
% -------------------------------------------------------------------------

\begin{document}

% -------------------------------------------------------------------------
% Title and authors
% -------------------------------------------------------------------------


\title{Invited Workshop: Empirical Investigation of Methods for
  Heterogeneity}

\author{\name Carlos Carvalho \email
  carlos.carvalho@mccombs.utexas.edu \\
  \addr Department of Statistics Data Science \\ Department of
  Information, Risk and Operations Management  \\
  The University of Texas at Austin\\
  Austin, TX 78712, USA
  %%
  \AND
  %%
  \name Jared Murray \email
  jared.murray@mccombs.utexas.edu \\
  \addr Department of
  Information, Risk and Operations Management  \\
  The University of Texas at Austin\\
  Austin, TX 78712, USA
  %%
  \AND
  %
  \name Spencer Woody  \email
  spencer.woody@utexas.edu \\
  \addr Department of Statistics and Data Science  \\
  The University of Texas at Austin\\
  Austin, TX 78712, USA}



\maketitle

% -------------------------------------------------------------------------
% Abstract and keywords
% -------------------------------------------------------------------------

\begin{abstract}%   <- trailing '%' for backward compatibility of .sty file
  Psychohistory combines history, sociology, and mathematical
  statistics to make general predictions about the future behavior of
  very large groups of people. There are two axioms of psychohistory:
  (1) the population whose behaviour is modeled should be sufficiently
  large; (2) the population should remain in ignorance of the results
  of the application of psychohistorical analyses.  In this paper, we
  use psychohistory to predict the results of large observational
  studies.
\end{abstract}

\begin{keywords}
  Heterogeneous treatment effects 
\end{keywords}

% -------------------------------------------------------------------------
% -------------------------------------------------------------------------
% CONTENT
% -------------------------------------------------------------------------
% -------------------------------------------------------------------------



\section{Introduction}

Psychohistory depends on the idea that, while one cannot foresee the
actions of a particular individual, the laws of statistics as applied
to large groups of people can predict the general flow of future
events~\citep{asimov1951}.  As an analogy, consider a gas.  It is
difficult to predict the motion of a single molecule in a gas, but we
can predict the mass action of the gas to a high level of accuracy.
In this paper, we apply psychohistory to predict the results of
observational studies.  \citet{cochran1965} defined an observational
study as an empiric investigation in which ''...the objective is to
elucidate cause-and-effect relationships…[in which] it is not feasible
to use controlled experimentation, in the sense of being able to
impose the procedures or treatments whose effects it is desired to
discover, or to assign subjects at random to different procedures.''


%%% Local Variables:
%%% mode: latex
%%% TeX-master: "../main"
%%% End:

...

% Acknowledgements should go at the end, before appendices and references

% -------------------------------------------------------------------------
% Acknowledgement
% -------------------------------------------------------------------------

\acks{We would like to acknowledge support for this project from
  Obi-Wan Kenobi of the Galactic Republic.}

% -------------------------------------------------------------------------
% Appendices
% -------------------------------------------------------------------------

% Manual newpage inserted to improve layout of sample file - not
% needed in general before appendices/bibliography.

\newpage

\appendix


\section{Proof of main theorem}
\label{app:theorem}

In this appendix we prove the following theorem from
Section~6.2:


%%% Local Variables:
%%% mode: latex
%%% TeX-master: "../..main"
%%% End:



% -------------------------------------------------------------------------
% Bibliography
% -------------------------------------------------------------------------

\vskip 0.2in
\bibliography{main}

\end{document}

%%% Local Variables:
%%% mode: latex
%%% TeX-master: t
%%% End:
